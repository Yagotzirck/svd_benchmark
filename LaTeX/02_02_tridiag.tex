\section{Tridiagonalizzazione della matrice}
Per brevità d'annotazione, d'ora in poi faremo riferimento alla matrice 
simmetrica (semi)definita positiva $A A^T$ (o $A^T A$, a seconda dei casi) con 
cui si sta lavorando chiamandola $B$.

Rimandando i dettagli all'apposita sezione, il primo passo da effettuare per 
accelerare la convergenza dell'algoritmo QR consiste nell'effettuare una 
trasformazione di $B$ in un'altra matrice $C$ trattabile più facilmente, tale 
che
\begin{itemize}
	\item Gli autovalori restino invariati: $\sigma(C) = \sigma(B)$;
	
	\item La forma semplificata di $C$ rimanga preservata per ogni iterazione 
dell'algoritmo QR (ovvero, la forma di $C$ dev'essere un'\textbf{invariante 
QR}).
\end{itemize}

\subsection{Forma di Hessenberg}
L'invariante QR che vogliamo ottenere nella matrice trasformata $C$ è la 
\textbf{forma di Hessenberg}, vale a dire una matrice triangolare superiore che 
ammette elementi non nulli nella prima sottodiagonale.

A tale scopo, si esegue una fattorizzazione di $B$ riconducibile alla 
fattorizzazione QR tramite inversioni di Householder, ma con la differenza che 
ad ogni iterazione $i$ non si modifica la colonna $\underline{b}_i$ azzerando 
tutti gli elementi sotto l'elemento diagonale $b_{ii}$, bensì \textbf{azzerando 
tutti gli elementi sotto l'elemento nella sottodiagonale $b_{i+1,i}$}.

Al fine di preservare gli autovalori, occorre inoltre effettuare una 
\textbf{trasformazione di similitudine}; ovvero, se $H_i$ è l'inversione di 
Householder all'iterazione $i$ tale che
\begin{equation}\label{householder_iteration}
H_i \cdot \underline{b}_{i}^{(i)} = 
\begin{bmatrix}
	b_{1i}^{(i)} \\
	b_{2i}^{(i)} \\
	\vdots \\
	b_{ii}^{(i)} \\ 
	\textcolor{red}{k_i} \\
	\textcolor{red}{0} \\ 
	\textcolor{red}{\vdots} \\ 
	\textcolor{red}{0}
\end{bmatrix}
\,.
\end{equation}

Invece di limitarcisi all'operazione $B^{(i+1)} = H_i \cdot B^{(i)}$ come nella 
fattorizzazione QR, occorre effettuare la trasformazione di similitudine
\begin{equation*}
B^{(i+1)} = H_i \cdot B^{(i)} \cdot H_{i}^T \,,
\end{equation*}
che, dal momento che la matrice elementare di Householder è simmetrica, si può 
semplificare come
\begin{equation}\label{hess_similarity_iteration}
B^{(i+1)} = H_i \cdot B^{(i)} \cdot H_{i} \,.
\end{equation}

Dal momento che $H_i$ è in realtà una matrice orlata, ovvero una matrice 
identità in cui è stata "inserita" la matrice di Householder $\hat{H}_i$ in 
basso a destra, il secondo prodotto a destra per $H_i$ non modificherà gli 
elementi sotto gli elementi sottodiagonali $\underline{b}_{i+1,i}$ 
che sono stati azzerati fino all'iterazione corrente $i$.

Una prima stesura dell'algoritmo è quindi
\begin{programma}
function [B,H] = hessenberg(B)
    size_B = size(B,1);
    n = size_B - 2;
    H = eye(size_B);
    I = eye(size_B);

    for i = 1:n
        x = B(i+1:end, i);
        H_hat = Householder(x);
        H_curr = I;
        H_curr(i+1:end, i+1:end) = H_hat;

        B = H_curr * B * H_curr;

        H = H * H_curr;
    end
end
\end{programma}

La funzione restituisce, oltre alla matrice di Hessenberg $C$, anche la matrice 
$H$ delle inversioni di Householder accumulate, tale che
\begin{equation}\label{B_C_similarity}
H C H^T = B \,.
\end{equation}

Tale relazione tornerà utile più avanti.

\subsection{$C$ è matrice tridiagonale}
Si osservi che nella \eqref{hess_similarity_iteration}, il prodotto per $H_i$ a 
sinistra va a modificare l'i-esima colonna di $B^{(i)}$ nel modo esplicitato 
nella \eqref{householder_iteration}, ed \textbf{essendo $B$ simmetrica}, il 
prodotto per $H_i$ a destra va a modificare l'i-esima riga di $B^{(i)}$ in 
maniera totalmente analoga (dato che l'i-esima riga è uguale alla trasposta 
dell'i-esima colonna):
\begin{equation*}
(\underline{b}_{i}^{(i)})^T \cdot H_i = 
\begin{bmatrix}
	b_{1i}^{(i)} &
	b_{2i}^{(i)} &
	\cdots &
	b_{ii}^{(i)} &
	\textcolor{red}{k_i} &
	\textcolor{red}{0} &
	\textcolor{red}{\cdots} &
	\textcolor{red}{0}
\end{bmatrix}
\,.
\end{equation*}

Ciò implica che alla fine della trasformazione, $C$ non è solo in forma di 
Hessenberg: è anche tridiagonale, con la sopradiagonale identica alla 
sottodiagonale (dal momento che stiamo lavorando con matrici reali simmetriche, 
non dobbiamo preoccuparci dei complessi coniugati). \\
Questo farà ancora più comodo nell'implementazione dell'algoritmo QR: dato che 
anche la simmetria di una matrice è un'invariante QR, ne segue che anche la 
\textbf{tridiagonalità è un'invariante QR}, e ciò darà modo di implementare 
diverse ottimizzazioni all'algoritmo, oltre a renderlo numericamente più 
stabile.

\subsection{Ottimizzazioni dell'algoritmo}
Partendo dalla prima bozza dell'algoritmo, possiamo apportare le seguenti 
ottimizzazioni:

\subsubsection{Sfruttamento del rango 1 della matrice di Householder}
Senza scendere troppo nel dettaglio, la matrice elementare di Householder nella 
sua versione calcolata senza far uso di radici quadrate è definita come $H = I - 
\frac{1}{\beta}vv^T$; piuttosto che calcolare $H$ esplicitamente ($O(n^2)$) per 
poi usarla per eseguire prodotti matriciali ($O(n^3)$), è più conveniente 
effettuare direttamente la moltiplicazione per le componenti di $H$:

\noindent	
\fbox{
\begin{minipage}[t]{0.45\textwidth}
	\centering $\mathbf{H \cdot A}$ \\
	\vspace{-20pt}
	\begin{align*}
	&H \cdot A = \\
	&(I - \frac{1}{\beta}vv^T) \cdot A = \\
	&A - \frac{1}{\beta}vv^T \cdot A = \\
	&A - v\cdot \frac{1}{\beta}(v^T \cdot A)
\end{align*}
\end{minipage}
}
\hfill
\fbox{
\begin{minipage}[t]{0.45\textwidth}
	\centering $\mathbf{A \cdot H}$ \\
	\vspace{-20pt}
	\begin{align*}
	&A \cdot H = \\
	&A \cdot (I - \frac{1}{\beta}vv^T) = \\
	&A - \frac{1}{\beta}A \cdot vv^T = \\
	&A - \frac{1}{\beta}(A \cdot v) \cdot v^T
\end{align*}
\end{minipage}
}

In questo modo, si riduce la complessità da un ordine di $O(n^3)$ ad un ordine 
di $O(n^2)$: nell'ultima derivazione di entrambe le forme $HA, AH$, abbiamo 
all'interno delle parentesi un \textbf{prodotto matrice-vettore} di complessità 
$O(n^2)$, seguito da una \textbf{scalatura degli elementi del vettore 
risultante} per $\frac{1}{\beta}$ di complessità $O(n)$, a cui seguono un 
\textbf{outer product} ed una \textbf{sottrazione matriciale}, entrambe di 
complessità $O(n^2)$.

Tali ottimizzazioni sono state implementate nella classe 
\href{https://github.com/Yagotzirck/svd_benchmark/blob/main/src/Householder.m}{Householder.m}, 
la quale restituirà un oggetto che tra le altre cose simula a livello sintattico 
il classico prodotto matriciale, astraendo i dettagli implementativi definiti 
all'interno della classe.

\subsubsection{Sfruttamento della simmetria}
Dal momento che per ogni iterazione $i$ la riga $i$ corrisponde alla trasposta 
della colonna $i$, possiamo applicare la matrice di Householder per azzerare gli 
elementi della colonna $i$ che stanno sotto l'elemento nella sottodiagonale, 
replicare tale colonna nella riga $i$ trasponendola, per poi applicare la 
trasformazione di similitudine alla sottomatrice ottenuta rimuovendo la riga e 
la colonna $i$ da $B$, in modo che l'iterazione successiva lavori su tale 
sottomatrice piuttosto che sull'intera matrice $B$. \\

Tali ottimizzazioni sono state implementate nella funzione 
\href{https://github.com/Yagotzirck/svd_benchmark/blob/main/src/tridiag.m}{tridiag.m}.

