\chapter{Fattorizzazione SVD}

\section{SVD standard}
Data una qualunque matrice
\begin{equation*}  
A \in \mathbb{C}^{m \times n},\quad k = \rank(A)
\end{equation*}
è sempre possibile effettuarne una fattorizzazione SVD che consente di scomporla 
in tre matrici $U, \Sigma, V$:
\begin{equation}\label{SVD_standard}
A = U \Sigma V^*
\end{equation}
dove
\begin{itemize}
	\item $U \in \mathbb{C}^{m \times m},\quad U^*U = UU^*=I_{m}$, le cui colonne 
sono
	definite come i \textbf{vettori singolari sinistri} di $A$;
	
	\item $\Sigma \in \mathbb{R}^{m\times n}$ è una matrice con $k$ elementi reali
	positivi lungo la diagonale (detti \textbf{valori singolari di $A$}), e zeri 
altrove;
	
	\item $V \in \mathbb{C}^{n \times n}, \quad V^*V = VV^*=I_{n}$, le cui colonne 
sono
	definite come i \textbf{vettori singolari destri} di $A$.
\end{itemize}  

Tale fattorizzazione è chiamata \textbf{fattorizzazione SVD piena / standard}.

\newpage
\section{SVD compatta}
Si osservi che al fine di ricostruire $A$, solo i primi $k$ vettori singolari 
sinistri e destri sono necessari; i restanti vettori
\begin{equation*}
	\underline{u}_{i} \in U, i = k+1 \cdots m
\end{equation*}  

\begin{equation*}
	\underline{v}_{i} \in V, i = k+1 \cdots n
\end{equation*}

sono vettori creati artificialmente al fine di ottenere due matrici ortogonali 
(o unitarie, se $A \in \mathbb{C}^{m \times m}$), creandoli in maniera casuale 
per poi ortonormalizzarli rispetto ai restanti vettori con un algoritmo come 
Householder o Modified Gram-Schmidt.

Si verifica facilmente che i vettori singolari "artificiali" non influiscono 
nella ricostruzione di $A$ dal fatto che in \eqref{SVD_standard}, essendo 
$\Sigma$ matrice diagonale con solamente i primi $k$ elementi sulla diagonale 
non nulli:
\begin{itemize}
	\item Se consideriamo $U \Sigma$, vengono scalate le prime $k$ colonne di $U$ 
ed azzerate le ultime $m-k$ colonne;
	
	\item Se consideriamo $\Sigma V^*$, vengono scalate le prime $k$ righe di $V^*$ 
ed azzerate le ultime $n-k$ righe (che equivale a dire che vengono scalate le 
prime $k$ colonne di $V$ ed azzerate le ultime $n-k$ colonne).
\end{itemize}

Se quindi definiamo le tre matrici
\begin{equation*}  
U_{1} = \begin{bmatrix}
\underline{u}_1 & \underline{u}_2 & \cdots & \underline{u}_k
\end{bmatrix}
\in \mathbb{C}^{m \times k},
\quad U^*U = I_k
\end{equation*}

\begin{equation*}  
\Sigma_{k} = \begin{bmatrix}
\sigma_1 & & & \\
& \sigma_2 & & \\
& & \ddots & \\
& & & \sigma_k
\end{bmatrix} \in \mathbb{R}^{k \times k},
\quad \sigma_i \in \mathbb{R} \wedge \sigma_i > 0, \quad i = 1 \cdots k
\end{equation*}

\begin{equation*}  
V_{1} = \begin{bmatrix}
\underline{v}_1 & \underline{v}_2 & \cdots & \underline{v}_k
\end{bmatrix}
\in \mathbb{C}^{n \times k},
\quad V^*V = I_k
\end{equation*}

otteniamo la \textbf{fattorizzazione SVD compatta} di $A$:
\begin{equation}\label{SVD_compact}
A = U_1 \Sigma_k {V_1}^*
\end{equation}

Chiamata anche \textbf{full-rank SVD} in quanto tutt'e tre le matrici hanno 
rango pieno k (\textit{$U_1$ e $V_1$ sono a rango pieno di colonna, mentre 
$\Sigma_k$ è quadrata a rango pieno}).


\section{Economy-sized SVD}
\label{sec:SVD_econ}
Stabilire il rango esatto $k$ di una matrice è impossibile da un punto di vista 
numerico, dal momento che a causa di errori relativi alla precisione di macchina 
si ottiene raramente il valore esatto 0 per i valori singolari $\sigma_i, \quad 
i > k$.

Infatti, quando si calcola il rango di una matrice se ne sta in realtà 
calcolando il rango numerico (o \textbf{$\epsilon$-rank}), in cui si sceglie una 
soglia $\epsilon$ e si considerano nulli tutti i $\sigma_i < \epsilon$.

L'affidabilità di tale approccio è però inversamente proporzionale al 
condizionamento della matrice sotto analisi; a titolo d'esempio, la teoria ci 
dice che la matrice di Hilbert è a rango pieno, tuttavia se eseguiamo le 
seguenti istruzioni su MATLAB:

\begin{programma}
H = hilb(100);
rank(H)
\end{programma}

Il risultato sarà 18, e non 100 come atteso.

Per le ragioni esplicate sopra, data una matrice $A \in \mathbb{C}^{m \times 
n}$, nella pratica viene eseguita una fattorizzazione SVD simile alla SVD 
compatta \eqref{SVD_compact}, ma con la differenza che al posto di
\begin{equation*}
k = \rank(A)
\end{equation*}

Si pone invece
\begin{equation*}
k = \min(m,n) \geq \rank(A)
\end{equation*}

Questo perchè il rango per righe di una matrice è uguale al suo rango per 
colonne, e di conseguenza è limitato superiormente dal minimo valore tra i due 
(numero di righe se la matrice è grassa, numero di colonne se la matrice è 
alta).

Tale fattorizzazione prende il nome di \textbf{Economy-sized SVD}, usata ad es. 
dalla funzione \textit{svd()} di MATLAB se viene specificato il parametro 
\textit{'econ'}.
