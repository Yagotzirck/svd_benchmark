\chapter{Conclusioni}
In questa relazione abbiamo implementato un algoritmo di fattorizzazione 
SVD reinterpretandolo come calcolo degli autovalori ed autovettori, sfruttando 
il legame che mette in relazione valori e vettori singolari di una generica 
matrice $A$ con autovalori ed autovettori delle matrici $A^T A$ e $A A^T$ da 
essa derivate.

Con una serie di test, abbiamo inoltre verificato che nonostante l'approccio 
usato sia corretto da un punto di vista teorico ed algebrico, una sua 
implementazione da usare in contesti pratici è poco conveniente da un punto di 
vista numerico, per i seguenti motivi:
\begin{enumerate}
	\item \textbf{Inefficienza}: Il calcolo preliminare della matrice $B = A^T A$ 
oppure $B = A A^T$ ha complessità di un ordine di grandezza pari a $O(n^3)$, che 
ha un peso non indifferente per matrici molto grandi;
	
	\item \textbf{Inaccuratezza e propagazione degli errori}: una volta trovata la 
prima matrice dei vettori singolari ed i valori singolari tramite l'algoritmo 
QR, quando si calcola la seconda matrice dei vettori singolari ricavandola da 
$A$ si avrà una perdita di ortonormalità tra i suoi vettori colonna direttamente 
proporzionale a $\kappa_2(A)$.
	
	Si potrebbe pensare di applicare l'algoritmo QR due volte e calcolare 
separatamente $U,V$ per risolvere il problema, ma oltre ad esacerbare il 
problema dell'inefficienza menzionato al punto 1), non è più garantito il fatto 
che $U \Sigma V^T = A$: ad esempio, i valori singolari ricavati da $A A^T$ 
potrebbero non essere congrui con quelli ricavati da $A^T A$.
	
	Inoltre, dato che $\kappa_2(B) = (\kappa_2(A))^2$, l'errore assoluto dei valori 
singolari calcolati risulta essere elevato al quadrato rispetto all'errore
assoluto dei valori singolari calcolati con \texttt{svd()}, che lavora
direttamente su $A$;
	
	\item \textbf{Overflow e underflow}: Con matrici aventi valori singolari molto 
grandi e/o prossimi allo zero, il calcolo di tali valori potrebbe andare 
rispettivamente in overflow o in underflow, impedendo così un calcolo corretto 
di tali valori singolari.
\end{enumerate}

Per i motivi sopra menzionati, la fattorizzazione SVD viene implementata tramite 
algoritmi che lavorano direttamente su $A$ basati sull'algoritmo di 
Golub/Kahan/Reinsch che ne effettua una bidiagonalizzazione come passo 
preliminare, ma i cui dettagli implementativi trascendono lo scopo di questa 
relazione.

