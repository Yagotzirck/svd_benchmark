\section{Algoritmo QR per matrici tridiagonali simmetriche}
Una volta ottenuta la matrice tridiagonale simmetrica $C$, possiamo procedere a 
calcolarne autovalori ed autovettori tramite l'algoritmo QR.

Nella sua forma base, l'algoritmo QR è rappresentabile nel seguente modo:

\begin{programma}
function [A,U] = eig_qr(A)
n = size(A,1);
U = eye(n);

while A ~= triu(A)
	[Q,R] = qr(A);
	A = R * Q;
	U = U * Q;
end

end
\end{programma}

Invocando quindi l'algoritmo nel seguente modo:
\begin{programma}
	[T,X] = eig_qr(A);
\end{programma}
Se l'algoritmo converge, restituirà due matrici $T,X$ con le quali è possibile 
ricostruire la \textbf{forma di Schur} di $A$:
\begin{equation*}
	T = X^* A X
\end{equation*}

Dove $X$ è matrice unitaria e $T$ è matrice triangolare superiore. 
\subsection{$A = A^* \Rightarrow $ Fattorizzazione spettrale}
Inoltre, se $A$ è Hermitiana, si può verificare facilmente che anche $T$ lo è, 
ma la combinazione delle proprietà di matrice Hermitiana e triangolare superiore 
implica che $T$ sia una matrice diagonale con elementi reali sulla diagonale, 
che a sua volta implica che le matrici $T,X$ forniscono la 
\textbf{fattorizzazione spettrale} di $A$:
\begin{equation*}
	A = X T X^*
\end{equation*}

Ovvero:
\begin{itemize}
	\item $\diag(T) = \sigma(A)$;
	\item Le colonne di $X$ sono gli autovettori di $A$.
\end{itemize}

\newpage
\subsection{Ottimizzazioni dell'algoritmo}
Vediamo ora in che modo possiamo migliorare l'algoritmo, sfruttando il fatto di 
lavorare con la matrice $C\in \mathbb{R}^{n \times n}$ tridiagonalizzata.

\subsubsection{Fattorizzazione QR}
Dal momento che gli unici elementi non nulli nel triangolo inferiore sono gli 
$n-1$ elementi nella sottodiagonale, possiamo effettuare la fattorizzazione QR 
azzerando selettivamente tali elementi (se sono maggiori di una certa soglia di 
tolleranza $\tau$) con \textbf{$n-1$ rotazioni di Givens}, riconducendoci ad una 
complessità lineare in ogni iterazione rispetto alla complessità cubica che si 
avrebbe invece applicando l'algoritmo di Householder. \\
Sono state inoltre applicate altre ottimizzazioni che sfruttano l'abbondanza di 
zeri nella matrice per il calcolo delle matrici $Q,R$, che sono state 
documentate nei commenti dell'implementazione nel file MATLAB 
\href{https://github.com/Yagotzirck/svd_benchmark/blob/main/src/qr_tridiag.m}{qr\_tridiag.m}.

\subsubsection{Prodotto $R \cdot Q$}
Sfruttando furbamente alcune proprietà come:
\begin{itemize}
	\item Le invarianti QR di $C$ matrice \textbf{simmetrica} e 
\textbf{tridiagonale};
	\item $R$ contenente una diagonale, due superdiagonali e zeri altrove;
	\item $Q$ matrice di Hessenberg
\end{itemize}

É possibile effettuare il prodotto $C = R \cdot Q$ ragionando solo sugli 
elementi sulla diagonale e sulla sottodiagonale di $C$ (la sopradiagonale è una 
copia della sottodiagonale), riducendolo così a
\begin{itemize}
	\item Due prodotti di Hadamard ed una somma vettoriale, per gli elementi sulla diagonale;
	\item Un prodotto di Hadamard, per gli elementi sulla sottodiagonale.
\end{itemize}
Essendo tutte operazioni che coinvolgono vettori estratti dalle diagonali, 
sottodiagonali e superdiagonali di $R$ e $Q$, si ottiene quindi una complessità 
con un ordine di grandezza pari a $O(n)$ invece di $O(n^3)$ che si avrebbe 
effettuando il prodotto matriciale pieno. 

Tale prodotto è stato implementato nella funzione ausiliaria \\
\href{https://github.com/Yagotzirck/svd_benchmark/blob/61d9bdf69b0431b83079c767cf2fc036bdc601a2/src/eig_tridiag.m#L60}{calc\_R\_times\_Q\_tridiag()}, 
presente nel file eig\_tridiag.m;

\subsubsection{Spectral shift}
Al fine di far convergere $C$ ad una matrice diagonale, è sufficiente verificare 
che ad ogni iterazione $i = n-1 \cdots 2$ l'elemento nella sottodiagonale 
$c_{i,i-1}$ sia inferiore ad una soglia prefissata $\tau$; se lo è, si procede 
ad analizzare il successivo elemento sottodiagonale, altrimenti si ripete il 
numero d'iterazioni dell'algoritmo QR necessarie affinchè $c_{i,i-1} < \tau$.

Inoltre, la teoria ci dice che la velocità di convergenza a 0 di ogni elemento 
sottodiagonale $c_{i,i-1}$ in relazione al numero di iterazioni $k$ è data dalla 
$k$-esima potenza del rapporto tra autovalori adiacenti:
\begin{equation*}
	\left( \frac{\lambda_{i+1}}{\lambda_i} \right)^k
\end{equation*}

\textcolor{red}{
Digressione: Nel nostro caso, anche se $\lambda_{i+1} = \lambda_i$ si arriva 
comunque alla convergenza; infatti, la non-convergenza si verifica quando si 
hanno autovalori di modulo uguale ma di segno opposto (oppure un autovalore è 
complesso coniugato dell'altro), ma dal momento che $C$ è (semi)definita 
positiva, si ha che
\begin{equation*}
	\lambda_i \geq 0 \wedge \lambda_i \in \mathbb{R}, \quad i = 1 \cdots n 
\end{equation*}
Il che implica che
\begin{equation*}
	|\lambda_{i+1}| = |\lambda_i| \Rightarrow \lambda_{i+1} = \lambda_i
\end{equation*}
}

L'idea è quindi quella di aumentare ad ogni iterazione il divario tra numeratore 
e denominatore affinchè $c_{i,i-1}$ converga a zero più velocemente, tramite una 
\textbf{traslazione dello spettro} di $C$ utilizzando $\tau_k = c_{i,i}$ come 
approssimazione di $\lambda_{i+1}$:
\begin{equation*}
	\left( \frac{\lambda_{i+1} - \tau_k}{\lambda_i - \tau_k} \right)^k \ll \left( 
\frac{\lambda_{i+1}}{\lambda_i} \right)^k
\end{equation*}

L'algoritmo QR diventa quindi:
\begin{align*}
&for \; k=0,1,\cdots \\
&	[Q_k, R_k] = qr(A^{(k)} - \tau_k \cdot I) \\
&	A^{(k+1)} = Q_k \cdot R_k + \tau_k \cdot I \\
\end{align*}

Tale tecnica prende il nome di \textbf{Rayleigh quotient shift}, e sebbene sia 
relativamente semplice aiuta ad accelerare la convergenza rispetto alla versione 
senza traslazione \textcolor{red}{(per concludere la digressione fatta 
precedentemente, se si sta invece lavorando con matrici per le quali non è 
garantita la convergenza, è più opportuno passare all'utilizzo di una 
traslazione più sofisticata come il \textbf{Wilkinson shift})}.

Ciò funziona in quanto si può verificare facilmente che la relazione di 
similitudine tra la successione di matrici resta valida:
\begin{equation*}
	A^{(k+1)} \sim A ^{k}, \quad k=0,1,\cdots
\end{equation*}

\subsubsection{Deflazione}
Una volta azzerato l'elemento sottodiagonale $c_{i,i-1}$, si ha che
\begin{itemize}
	\item $c_{ii} = \lambda_n$;
	\item $\underline{u}_i$ è l'autovettore associato a $\lambda_n$.
\end{itemize}

Dal momento che l'i-esimo autovalore ed autovettore si sono stabilizzati, 
piuttosto che continuare a lavorare sulla matrice $C \in \mathbb{R}^{i \times 
i}$ di ordine $i$, è più opportuno effettuarne una \textbf{deflazione} 
riducendola ad un ordine $i-1$ rimuovendo l'ultima riga e colonna, al fine di 
ridurre la complessità computazionale nelle iterazioni successive dell'algoritmo 
QR.

Tale accorgimento (assieme a quello relativo alla traslazione dello spettro) è 
stato implementato nel file 
\href{https://github.com/Yagotzirck/svd_benchmark/blob/main/src/eig_tridiag.m}{eig\_tridiag.m}, 
nell'omonima funzione.
