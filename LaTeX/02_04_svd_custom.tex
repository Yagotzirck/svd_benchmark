\section{Calcolo dell'SVD}
Mettendo assieme tutto quanto è stato visto finora, date le matrici
\begin{itemize}
	\item $A$: Matrice reale generica, di cui vogliamo effettuare la 
fattorizzazione SVD;
	\item $B$: Matrice (semi)definita positiva, ottenuta dal prodotto $A A^T$ 
oppure $A^T A$;
	\item $C$: Matrice tridiagonale simmetrica, ricavata da $B$ e tale che $C \sim 
B$;
	\item $H$: Matrice ortogonale, con la quale vale la relazione di similitudine 
\eqref{B_C_similarity}
\end{itemize}

Al fine di ottenere autovalori ed autovettori di $B$, il primo passo consiste 
nel chiamare la funzione
\begin{programma}
[X,e] = eig_tridiag(C);
\end{programma}

 Si ha quindi che
 \begin{itemize}
 	\item $\sigma(C) = e \wedge C \sim B \Rightarrow \sigma(B) = e$;
 	\item X è la matrice degli autovettori di $C$, per la quale, data la matrice 
$\Lambda = \diag(e)$, vale la fattorizzazione spettrale $X^T C X = \Lambda$.
 \end{itemize}

Partendo dalla \eqref{B_C_similarity}, si ha che $C = H^T B H$; sostitendola 
nella fattorizzazione spettrale citata sopra, si ottiene 

 \begin{equation*}
	(\textcolor{red}{H X})^T B (\textcolor{red}{H X}) = \Lambda
\end{equation*}
Ovvero, $\textcolor{red}{H X}$ è la matrice degli autovettori di $B$, il che 
implica a sua volta che

\begin{itemize}
	\item $B = A A^T \Rightarrow H X = U$ (vettori singolari sinistri di $A$);
	\item $B = A^T A \Rightarrow H X = V$ (vettori singolari destri di $A$).
\end{itemize}

I passi per concludere il calcolo dell'SVD sono i seguenti:
\begin{enumerate}
	\item Ordinamento degli autovalori nel vettore $e$, in ordine non crescente;
	\item Ordinamento dei vettori singolari in $B$, usando gli stessi indici di 
ordinamento usati per $e$ al fine di mantenere l'associazione tra valori 
singolari e corrispondenti vettori singolari;
	\item Calcolo della radice quadrata degli elementi in $e$, per passare da 
autovalori di $B$ a valori singolari di $A$;
	\item Calcolo dell'altra matrice dei vettori singolari, seguendo quanto 
spiegato nella \autoref{subsec:SVD_calc}.
\end{enumerate}

\noindent
I passaggi descritti in questa sezione sono stati implementati nel file 
\href{https://github.com/Yagotzirck/svd_benchmark/blob/main/src/svd_custom.m}{svd\_custom.m}.